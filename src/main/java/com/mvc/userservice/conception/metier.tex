\documentclass[11pt,a4paper]{article}
\usepackage[utf8]{inputenc}
\usepackage[T1]{fontenc}
\usepackage[french]{babel}
\usepackage{lmodern}
\usepackage{geometry}
\geometry{margin=2cm}
\usepackage{enumitem}
\usepackage{xcolor}
\usepackage{tikz}
\usepackage{tikz-uml}
\usepackage{hyperref}
\usepackage{fancyhdr}
\usepackage{lastpage}

\pagestyle{fancy}
\fancyhf{}
\rhead{\textbf{User Service Microservice}}
\lhead{E-Banking 3.0}
\cfoot{Page \thepage\ / \pageref{LastPage}}

\title{\textbf{Documentation Métier et Technique\\Microservice User Service}\\E-Banking 3.0 (Microservices modernes)}
\author{}
\date{28 décembre 2025}

\begin{document}

\maketitle
\thispagestyle{fancy}

\section{Contexte Métier Global du Projet E-Banking 3.0}

Le projet \textbf{E-Banking 3.0} vise à moderniser l'application bancaire existante en adoptant une architecture \textbf{microservices}, avec les objectifs suivants~:

\begin{itemize}
    \item Améliorer l'\textbf{expérience utilisateur (UX)} grâce à un frontend SPA (Angular) et des requêtes agrégées via \textbf{GraphQL}.
    \item Découper la plateforme en \textbf{microservices indépendants} pour une meilleure scalabilité et des tests isolés.
    \item Conserver l'\textbf{interopérabilité} avec les systèmes bancaires legacy via \textbf{SOAP}.
    \item Ajouter des services innovants~: portefeuille crypto, assistant IA, paiement biométrique, dashboards interactifs.
    \item Respecter strictement la \textbf{RGPD} (consentements, droit à l'effacement, journaux d'accès).
\end{itemize}

Le \textbf{microservice user-service} est au cœur de cette architecture~: il gère \textbf{les profils utilisateurs, la vérification d'identité (KYC)} et \textbf{les consentements RGPD}.

\section{Rôle de Chaque Classe / Entité Principale}

\begin{description}
    \item[User] \hfill \\
        Entité centrale représentant un utilisateur de la banque (client, agent ou admin).\\
        \textbf{Rôle~:} Stocker les données personnelles et métier (nom, email, téléphone, adresse, rôle, statut KYC).\\
        Contient le \texttt{keycloakId} (sub du JWT) pour liaison immuable avec Keycloak (source d'authentification).

    \item[ConsentType] \hfill \\
        Entité dynamique représentant un \textbf{type de consentement RGPD} (ex. : marketing email, biométrie, partage données).\\
        \textbf{Rôle~:} Permettre à un administrateur d'ajouter ou désactiver des types de consentement sans redéploiement.\\
        Champs clés~: \texttt{code}, \texttt{name}, \texttt{isActive}, \texttt{nbr} (compteur d'activations globales).

    \item[Consent] \hfill \\
        Représente le consentement \textbf{donné par un utilisateur spécifique} pour un type donné.\\
        \textbf{Rôle~:} Historiser si l'utilisateur a accepté (\texttt{isOk = true}) ou révoqué un consentement.\\
        Relation ManyToOne avec \texttt{ConsentType} (instances partagées).

    \item[KycDocument] \hfill \\
        Représente un document soumis dans le cadre de la vérification d'identité (KYC).\\
        \textbf{Rôle~:} Stocker le type, le statut et le chemin du fichier uploadé.\\
        Relation OneToMany avec \texttt{User}.

    \item[UserRole] (enum) \hfill \\
        Définit les rôles possibles~: \texttt{ROLE\_CLIENT}, \texttt{ROLE\_AGENT}, \texttt{ROLE\_ADMIN}.\\
        Dupliqué depuis Keycloak pour performance et autonomie du service.

    \item[KycStatus] (enum) \hfill \\
        Statut global KYC de l'utilisateur~: \texttt{PENDING}, \texttt{IN\_REVIEW}, \texttt{APPROVED}, \texttt{REJECTED}.\\
        Recalculé automatiquement après chaque action sur un document.

    \item[KycDocumentType / KycDocumentStatus] (enums) \hfill \\
        Types et statuts des documents KYC (ex. : pièce d'identité, selfie, approuvé/rejeté).
\end{description}

\section{Relations entre les Classes}

\begin{tikzpicture}
\begin{umlpackage}{com.mvc.userservice.entity}
    \umlclass[x=0,y=0]{User}{
        - id : UUID\\
        - keycloakId : UUID\\
        - username : String\\
        - email : String\\
        - role : UserRole\\
        - kycStatus : KycStatus\\
        ...
    }{
    }

    \umlclass[x=-5,y=-4]{Consent}{
        - id : UUID\\
        - isOk : boolean\\
        - grantedAt : LocalDateTime\\
        - revokedAt : LocalDateTime
    }{
        + revoke()
    }

    \umlclass[x=5,y=-4]{KycDocument}{
        - id : UUID\\
        - documentType : KycDocumentType\\
        - status : KycDocumentStatus\\
        - pathToDocument : String
    }{
    }

    \umlclass[x=0,y=-4]{ConsentType}{
        - id : UUID\\
        - code : String\\
        - name : String\\
        - isActive : boolean\\
        - nbr : int
    }{
    }

    \umlclass[x=0,y=4]{UserRole}{}{}
    \umlclass[x=5,y=4]{KycStatus}{}{}

    % Relations
    \umlaggreg[geometry=-|, mult=1]{User}{Consent}
    \umlaggreg[geometry=-|, mult=1]{User}{KycDocument}

    \umlassoc[geometry=--, mult1=*, mult2=1, anchor1=south, anchor2=north]{Consent}{ConsentType}

    \umlassoc[geometry=--, mult1=1, mult2=1]{User}{UserRole}
    \umlassoc[geometry=--, mult1=1, mult2=1]{User}{KycStatus}

\end{umlpackage}
\end{tikzpicture}

\section{Conclusion}

Le microservice \textbf{user-service} respecte pleinement les exigences du projet E-Banking 3.0~:
\begin{itemize}
    \item Séparation claire entre authentification (Keycloak) et données métier.
    \item Gestion dynamique des consentements RGPD.
    \item Processus KYC robuste avec recalcul automatique.
    \item Sécurité fine via JWT et rôles dupliqués.
    \item Prêt pour intégration via API Gateway GraphQL et Kafka.
\end{itemize}

Ce design garantit scalabilité, maintenabilité et conformité réglementaire.

\end{document}